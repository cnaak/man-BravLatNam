%----------------------------------------------------------------------------------------------%
%                                           Packages                                           %
%----------------------------------------------------------------------------------------------%
\usepackage{booktabs}
\usepackage[english]{babel}
\usepackage[squaren,cdot]{SIunits}
\usepackage{amsmath}
\usepackage{amssymb}
\usepackage{amsthm}
\usepackage{CormorantGaramond}
\usepackage{xspace}
%-----------------------------------------------------------------------------------------------
\usepackage[hyperindex,breaklinks]{hyperref} % Required for hyperlinks
%----------------------------------------------------------------------------------------------%
%                                           Commands                                           %
%----------------------------------------------------------------------------------------------%
\hypersetup{%
    hidelinks,
    colorlinks,
    breaklinks=true,
    urlcolor=color3,
    citecolor=color1,
    linkcolor=color1,
    bookmarksopen=false,
    pdftitle={Title},
    pdfauthor={Author}
}
%-----------------------------------------------------------------------------------------------
\setlength{\abovecaptionskip}{4pt}
\setlength{\columnsep}{5.5mm}
\setlength{\columnseprule}{0.2pt}
\setlength{\fboxrule}{0.4pt} % Width of the border around the abstract
%-----------------------------------------------------------------------------------------------
\definecolor{color1}{RGB}{0,0,90} % Color of the article title and sections
\definecolor{color2}{RGB}{0,20,20} % Color of the boxes behind the abstract and headings
\definecolor{color3}{RGB}{0,0,192} % Color of the article title and sections
%-----------------------------------------------------------------------------------------------
\newtheorem{theorem}{Theorem}
\newtheorem{definition}{Definition}
%-----------------------------------------------------------------------------------------------
\newcommand{\XXX}[1]{\relax}
%----------------------------------------------------------------------------------------------%
%                                           Metadata                                           %
%----------------------------------------------------------------------------------------------%
\makeatletter
\immediate\write18{datelog > \jobname.info}
\makeatother
%-----------------------------------------------------------------------------------------------
% Journal information
\JournalInfo{engrXiv}
\Archive{Compiled on \input{\jobname.info} -- Version 0
}
% Article title
\PaperTitle{Bravais Lattice Naming Schemes for Lattice-Boltzmann Methods}
\Authors{%
    C.~Naaktgeboren\textsuperscript{1$\star$},
    F.~N.~de Andrade\textsuperscript{2}
}
\affiliation{%
    \textsuperscript{1}%
    \textit{%
        Adjunct Professor.
        Universidade Tecnológica Federal do Paraná -- UTFPR, Câmpus Guarapuava.
        Grupo de Pesquisa em Ciências Térmicas.
}}
\affiliation{%
    \textsuperscript{2}%
    \textit{%
        Universidade Tecnológica Federal do Paraná -- UTFPR, Câmpus Guarapuava.
        Grupo de Pesquisa em Ciências Térmicas.
}}
\affiliation{%
    \textsuperscript{$\star$}%
    \textbf{Corresponding  author}: NaaktgeborenC$\cdot$PhD@gmail$\cdot$com
}
\Keywords{%
    Bravais lattice ---
    lattice-Boltzmann stencils ---
    lattice naming scheme ---
    method of prescribed abscissas ---
    space-filling lattices ---
    high-order lattices ---
    square lattices ---
    hexagonal lattices.
}
\newcommand{\keywordname}{Keywords}
\Highlights{%
    Proposes an unambiguous, more descriptive lattice naming scheme for LBM ---
    Lattice naming scheme is based on microscopic kinetic energy levels
}
\newcommand{\highlightname}{Highlights}
%-----------------------------------------------------------------------------------------------
\Abstract{%
    The lattice configuration is a central parameter on  Lattice-Boltzmann  (LB)  methods,  both
    from the theoretical and from the implementation standpoints. As LB methods have  progressed
    over the past decades, a variety of lattice configurations have been proposed and refered to
    according to a plurality  of  lattice  naming  schemes  that  usually  include  the  lattice
    \emph{velocity count} in their format. The  so-called  space-filling,  or,  Bravais  lattice
    types are of particular interest for hydrodynamic  and  hydro-thermal  applications  in  the
    sense that the resulting LB methods are capable of resolving convection in an exact fashion,
    thus avoiding the artifact of spurious numerical diffusion. The appearance  of  higher-order
    LB methods, that employ progressivelly broader velocity-space discretizations  by  means  of
    progressively more numerous lattice velocity sets, has led to the  appearance  of  ambiguous
    velocity-count-based lattice nomenclatures that often require further  information---usually
    the entire set of velocity vectors,  or,  conversely,  energy  levels---for  an  unambiguous
    description. This work proposes  a  concise  yet  unambiguous  naming  scheme  for  regular,
    space-filling (Bravais) lattices for LB methods based on \emph{lattice energy levels} in one
    to three Euclidean spatial dimensions for lattices of \emph{any} finite size.  The  proposed
    naming scheme is general enough as to  be  able  of  unambiguously  reffering  to  all  such
    lattices in the literature, under a uniform and common naming rule.
}
%-----------------------------------------------------------------------------------------------
