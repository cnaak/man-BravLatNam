%------------------------------------------------------------------------------------------------------------------------------%
%                                                          Title Page                                                          %
%------------------------------------------------------------------------------------------------------------------------------%

\thispagestyle{empty} % Removes page numbering from the first page
\flushbottom % Makes all text pages the same height
\maketitle % Print the title and abstract box


%------------------------------------------------------------------------------------------------------------------------------%
%                                                           License                                                            %
%------------------------------------------------------------------------------------------------------------------------------%

\section*{License}

    \scriptsize\noindent%
    \begin{minipage}{\columnwidth}
        \centering\tt
        \includegraphics[height=6.0mm]{cc/by.pdf}\\[0.5\smallskipamount]
        {\scriptsize\url{https://creativecommons.org/licenses/by/4.0/}}
    \end{minipage}
    \normalsize


%------------------------------------------------------------------------------------------------------------------------------%
%                                                      Table Of Contents                                                       %
%------------------------------------------------------------------------------------------------------------------------------%

\tableofcontents


%------------------------------------------------------------------------------------------------------------------------------%
%                                                         Introduction                                                         %
%------------------------------------------------------------------------------------------------------------------------------%

\section{Introduction}

    Historically,   the   Lattice-Boltzmann   (LB)    method    had    its    origins    in    the    frame    of    Lattice-Gas
    Automata~\cite{1988-McNamaraGR+ZanettiG-PhysRevLett},     and     has     been     intensely     developed     since     its
    inception~\cite{2018-KrugerT+ViggenEM-Springer}. One important conceptual and implementational parameter of  LB  methods  is
    the    employed    \emph{lattice    stencil}---the    lattice    geometry,    velocity    set,    weights,     and     scale
    parameters~\cite{2013-HegeleJr+PhilippiPC-JSciComput,                                2013-MattilaKK+PhilippiPC-IntJModPhysC,
    2014-MattilaKK+PhilippiPC-SciWorldJ}. Both LB and LGA methods can be implemented on a variety of lattices, and  historically
    many such lattices (along with their corresponding stencils) have been developed.

    From early in the history of LGA and LB methods, it was realized that lattice  stencils  that  are  isotropic  to  a  higher
    velocity moment order are able to recover and thus reproduce macroscopic physical phenomena, such as the  Navier-Stokes,  in
    particular    the    momentum    equation,    more    accurately    and    with    more    isotropic    Taylor     expansion
    residues~\cite{1986-WolframS-JStatPhys}.

    During the transition period from LGA to LB methods, one can recover from the literature a brief period  of  varied  lattice
    naming, in which some authors have named lattices with acronyms after its  first  proposers  while  others  adopted  a  more
    geometrical  approach  to  naming  lattices.  Soon  after  this  inception  period,  seemingly  after  the  works  of   Qian
    et~al.~\cite{1991-QianYH+LallemandP-AdvKinTheoContMech, 1992-QianYH+LallemandP-EurophysLett}, the  LB  scientific  community
    began to  converge  towards  a  lattice  naming  scheme  based  on  the  lattice  \emph{Euclidean  dimensionality}  $d$  and
    \emph{velocity  count},  $b$---the  so-called  ``\texttt{DdQb}''  naming  scheme,  as  with  \texttt{D1Q3},   \texttt{D2Q9},
    \texttt{D3Q15}~\cite{1992-QianYH+LallemandP-EurophysLett}, etc. This seems to be the most prevalent lattice naming system to
    date, although notable exceptions appear long after ref~\cite{1991-QianYH+LallemandP-AdvKinTheoContMech} came out in 1991.

    Work in progress... Here's a citation~\cite{2016-PhilippiPC+MattilaKK-JBrazSocMechSci}.

