%------------------------------------------------------------------------------------------------------------------------------%
%                                                          Title Page                                                          %
%------------------------------------------------------------------------------------------------------------------------------%

\thispagestyle{empty} % Removes page numbering from the first page
\flushbottom % Makes all text pages the same height
\maketitle % Print the title and abstract box


%------------------------------------------------------------------------------------------------------------------------------%
%                                                           License                                                            %
%------------------------------------------------------------------------------------------------------------------------------%

\section*{License}

    \scriptsize\noindent%
    \begin{minipage}{\columnwidth}
        \centering\tt
        \includegraphics[height=6.0mm]{cc/by.pdf}\\[0.5\smallskipamount]
        {\scriptsize\url{https://creativecommons.org/licenses/by/4.0/}}
    \end{minipage}
    \normalsize


%------------------------------------------------------------------------------------------------------------------------------%
%                                                      Table Of Contents                                                       %
%------------------------------------------------------------------------------------------------------------------------------%

\tableofcontents


%------------------------------------------------------------------------------------------------------------------------------%
%                                                         Introduction                                                         %
%------------------------------------------------------------------------------------------------------------------------------%

\section{Introduction}

    Historically,   the   Lattice-Boltzmann   (LB)    method    had    its    origins    in    the    frame    of    Lattice-Gas
    Automata~\cite{1988-McNamaraGR+ZanettiG-PhysRevLett},     and     has     been     intensely     developed     since     its
    inception~\cite{2018-KrugerT+ViggenEM-Springer}. One important conceptual and implementational parameter of  LB  methods  is
    the    employed    \emph{lattice    stencil}---the    lattice    geometry,    velocity    set,    weights,     and     scale
    parameters~\cite{2013-HegeleJr+PhilippiPC-JSciComput,                                2013-MattilaKK+PhilippiPC-IntJModPhysC,
    2014-MattilaKK+PhilippiPC-SciWorldJ}. Both LB and LGA methods can be implemented on a variety of lattices, and  historically
    many such lattices (along with their corresponding stencils) have been developed.

    Some   LGA   lattices   were   named   with   \emph{acronyms}   after   its   first   proposers,   such   as   the   ``HPP''
    one~\cite{1986-FrischU+PomeauY-PhysRevLett},  after  Hardy,  de  Pazzis,   and   Pomeau~\cite{1973-HardyJ+PazzisO-JMathPhys,
    1976-HardyJ+PomeauY-PhysRevA},      or      geometry-based      \emph{acronyms},       such       as       the       ``HLG''
    one~\cite{1986-FrischU+PomeauY-PhysRevLett}, which stands for ``hexagonal lattice gas,'' later on referred to as the ``FHP''
    one~\cite{1987-FrischU+RivetJP-ComplexSyst}, after Frisch, Hasslacher, and Pomeau. Another  geometry-based  lattice  of  the
    time is the ``FCHC'' one~\cite{1987-FrischU+RivetJP-ComplexSyst}, which stands for ``face-centered-hypercubic''  model,  due
    to d'Humières, Lallemand, and Frisch, apud reference \{2\} from~\cite{1987-FrischU+RivetJP-ComplexSyst}.
    

    During the transition period from LGA to LB methods, one can recover from the literature a brief period  of  varied  lattice
    naming, in which some authors have named lattices with acronyms after its  first  proposers  while  others  adopted  a  more
    geometrical  approach  to  naming  lattices.  Soon  after  this  inception  period,  seemingly  after  the  works  of   Qian
    et~al.~\cite{1991-QianYH+LallemandP-AdvKinTheoContMech, 1992-QianYH+LallemandP-EurophysLett}, the  LB  scientific  community
    began to  converge  towards  a  lattice  naming  scheme  based  on  the  lattice  \emph{Euclidean  dimensionality}  $d$  and
    \emph{velocity  count},  $b$---the  so-called  ``\texttt{DdQb}''  naming  scheme,  as  with  \texttt{D1Q3},   \texttt{D2Q9},
    \texttt{D3Q15}~\cite{1992-QianYH+LallemandP-EurophysLett}, etc. This seems to be the most prevalent lattice naming system to
    date, although notable exceptions appear long after reference~\cite{1991-QianYH+LallemandP-AdvKinTheoContMech} came  out  in
    1991.

    The relationship between mesoscopic lattice \emph{symmetry} and resulting macroscopic description \emph{isotropy}  has  been
    established from early in the history of LGA methods~\cite{1973-HardyJ+PazzisO-JMathPhys, 1976-HardyJ+PomeauY-PhysRevA},  in
    two~\cite{1986-FrischU+PomeauY-PhysRevLett} and in three Euclidean dimensions, the latter requiring the lattice  to  include
    links  beyond  nearest  neighbors~\cite[pp.~473,490]{1986-WolframS-JStatPhys},  hence  particle  velocities   with   unequal
    magnitudes.

    % Seek for higher-order lattices (HOL)

    % HOL named in the DdQb scheme begin to display ambiguity

    % Reason: energy levels with similar velocity count

    % Look for velocity count integer sequences: square, hex 2D... 3D

