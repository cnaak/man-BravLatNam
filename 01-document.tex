%------------------------------------------------------------------------------------------------------------------------------%
%                                                          Title Page                                                          %
%------------------------------------------------------------------------------------------------------------------------------%

\thispagestyle{empty} % Removes page numbering from the first page
\flushbottom % Makes all text pages the same height
\maketitle % Print the title and abstract box


%------------------------------------------------------------------------------------------------------------------------------%
%                                                           License                                                            %
%------------------------------------------------------------------------------------------------------------------------------%

\section*{License}

    \scriptsize\noindent%
    \begin{minipage}{\columnwidth}
        \centering\tt
        \includegraphics[height=6.0mm]{cc/by.pdf}\\[0.5\smallskipamount]
        {\scriptsize\url{https://creativecommons.org/licenses/by/4.0/}}
    \end{minipage}
    \normalsize


%------------------------------------------------------------------------------------------------------------------------------%
%                                                      Table Of Contents                                                       %
%------------------------------------------------------------------------------------------------------------------------------%

\tableofcontents


%------------------------------------------------------------------------------------------------------------------------------%
%                                                         Introduction                                                         %
%------------------------------------------------------------------------------------------------------------------------------%

\section{Introduction}

    Historically,   the   Lattice-Boltzmann   (LB)    method    had    its    origins    in    the    frame    of    Lattice-Gas
    Automata~\cite{1988-McNamaraGR+ZanettiG-PhysRevLett},     and     has     been     intensely     developed     since     its
    inception~\cite{2018-KrugerT+ViggenEM-Springer}. One important conceptual and implementational parameter of  LB  methods  is
    the    employed    \emph{lattice    stencil}---the    lattice    geometry,    velocity    set,    weights,     and     scale
    parameters~\cite{2013-HegeleJr+PhilippiPC-JSciComput,                                2013-MattilaKK+PhilippiPC-IntJModPhysC,
    2014-MattilaKK+PhilippiPC-SciWorldJ}. Both LB and LGA methods can be implemented on a variety of lattices, and  historically
    many such lattices (along with their corresponding stencils) have been developed.

    %---------------------------------------------------------------------------------------------------------------------------
    \subsection{Early-years lattice designations}

    Some   LGA   lattices   were   named   with   \emph{acronyms}   after   its   first   proposers,   such   as    the    `HPP'
    one~\cite{1986-FrischU+PomeauY-PhysRevLett},  after  Hardy,  de  Pazzis,   and   Pomeau~\cite{1973-HardyJ+PazzisO-JMathPhys,
    1976-HardyJ+PomeauY-PhysRevA,  1987-SucciS-JPhysAMathGen},  or   geometry-based   \emph{acronyms},   such   as   the   `HLG'
    one~\cite{1986-FrischU+PomeauY-PhysRevLett}, which stands for `hexagonal lattice gas,' later on referred  to  as  the  `FHP'
    one~\cite{1987-FrischU+RivetJP-ComplexSyst,  1987-SucciS-JPhysAMathGen},  after  Frisch,  Hasslacher,  and  Pomeau.  Another
    geometry-based  lattice  of  the  time  is  the  `FCHC'  one~\cite{1987-FrischU+RivetJP-ComplexSyst},   which   stands   for
    `face-centered-hypercubic' model, due to d'Humières, Lallemand, and Frisch. Later on designations such  as  `FHP  +  3  rest
    particles' and `FCHC + 3 rest particles' also appeared~\cite{1991-BoonJP-PhysD}, as well as suffixes such as  `-I',  `-III',
    and   `-IV'   after   `FHP',   for   alternative   collision   rules~\cite{1991-AppertC+ZaleskiS-PhysD,   1991-BoonJP-PhysD,
    1991-ChenS+RoseH-PhysD}.

    LB   methods   adhered    to    LGA    lattice    nomenclature    in    its    inception    period,    as    witnessed    by
    reference~\cite{1988-McNamaraGR+ZanettiG-PhysRevLett}          in          1988          and          by          subsequent
    references~\cite{1989-HigueraFJ+JimenezJ-EurophysLett,         1989-HigueraFJ+SucciS-EurophysLett}         in          1989,
    by~\cite{1990-BenziR+VergassolaM-EurophysLett,   1990-BenziR+VergassolaM-NuclPhysB,    1990-CancelliereA+SucciS-PhysFluidsA,
    1990-VergassolaM+SucciS-EurophysLett}   in   1990,   and   by~\cite{1991-CornubertR+LevermoreD-PhysD,    1991-ErnstMH-PhysD,
    1991-FrischU-PhysD, 1991-GunstensenAK+ZanettiG-PhysRevA, 1991-SucciS+BenziR-PhysRevA} in 1991, to cite a few.

    It seems that Qian~\cite{1990-QianYH-Paris} (apud~\cite[p.~235]{1993-QianYH-JSciComput}) was the one to introduce, in  1990,
    the `\texttt{DdQb}' lattice naming scheme for LB methods---in which $d$ is the lattice \emph{Euclidean  dimensionality}  and
    $b$   is   the   lattice   \emph{velocity   count},   as    in    \texttt{D1Q3},    \texttt{D2Q9},    and    \texttt{D3Q15},
    etc.~\cite{1992-QianYH+LallemandP-EurophysLett}---that seems to be  the  most  prevalent  lattice  naming  system  to  date,
    although notable exceptions appear long after the paper~\cite{1991-QianYH+LallemandP-AdvKinTheoContMech} came out in 1991.

    As far as increasing lattice velocity counts go, the relationship between mesoscopic lattice \emph{symmetry}  and  resulting
    macroscopic   description   \emph{isotropy}   has   been   established    from    early    in    the    history    of    LGA
    methods~\cite{1973-HardyJ+PazzisO-JMathPhys, 1976-HardyJ+PomeauY-PhysRevA},  in  two~\cite{1986-FrischU+PomeauY-PhysRevLett}
    and  in  three  Euclidean   dimensions,   the   latter   requiring   the   lattice   to   include   links   beyond   nearest
    neighbors~\cite[pp.~473,490]{1986-WolframS-JStatPhys}, hence particle velocities with unequal magnitudes.

    Moreover, Koelman~\cite{1991-KoelmanJMVA-EurophysLett} had proposed matching discrete velocity moments up to a certain order
    $n$ with the $d$-dimensional continuous Boltzmann distribiution, since only those moments  influence  the  macroscopic  flow
    behaviour; such procedure would yield values for lattice velocity \emph{weights} $W_{\alpha}$---nowadays $w_i$. The proposed
    criteria   were   deemed   more   stringent   than   previously    well-known    symmetry    and    isotropy    requirements
    from~\cite{1986-WolframS-JStatPhys}, since it not only led to an isotropic macroscopic description, but also ensure pressure
    term independence from velocity terms of the Navier-Stokes description. Furthermore, a skewed  rectangular  9-speed  lattice
    with independent $a$ and $b$ axis lengths was proposed\footnote{That lattice was named  `face-centred  rectangular'  by  the
    author.}, whose weights exactly recover those of the well-known \texttt{D2Q9} lattice for $a = b$, over which  the  argument
    that valid weights `[...] \emph{can always be found by choosing a large enough set of} (lattice)  \emph{velocity  vectors\/}
    [...]'~\cite{1991-KoelmanJMVA-EurophysLett}.

    One    driving    application     for     increased     velocity     count     lattices     is     thermal     flows.     On
    reference~\cite{1993-AlexanderFJ+SterlingJD-PhysRevE} an \emph{unnamed}  2D,  hexagonal  (triangular),  13-velocity  lattice
    having velocity magnitudes of 0,  1,  and  2  lattice  units~\cite{1998-ChenS+DoolenGD-AnnuRevFluidMech}  was  employed  for
    adiabatic sound propagation and heat transfer Couette flow, whose results were shown to be in agreement  with  corresponding
    analytical solutions.

    Some `\texttt{nDmV}' lattices, with $n$ being the Euclidean space dimension and $m$  the  lattice  velocity  count,  namely,
    \texttt{1D5V}, \texttt{2D16V}, and \texttt{3D40V}, were introduced  in~\cite{1994-ChenY+AkiyamaM-PhysRevE}  for  shock  wave
    front structure and shear wave flow application cases.  The  \texttt{2D16V}  lattice  was  said  to  be  comprised  of  four
    \emph{sublattices}, a term that appeared in subsequent references, with each sublattice having 4 discrete velocities of same
    magnitude and forming adjacent right angles, which led to possibly multiple sublattices per lattice energy  level  $\epsilon
    \equiv 2e = \xi^2$, with $\xi$ being the microscopic (lattice)  velocity  magnitude,  and  $e$  the  corresponding  specific
    kinetic energy, as was the case with the $\epsilon = 1^2 + 2^2 = 5$ energy level of a square lattice, represented by  the  8
    discrete velocities $(\pm\{1,2\},\pm\{2,1\})$ in lattice units. This is in contrast to the seemingly prevailing current view
    of a multiple-velocity lattice being a single lattice entity, without explicit reference to sublattices.

    Most likely borrowing from mesh-based CFD methods, a study~\cite{1996-HeX+DemboM-JComputPhys} has proposed  a  LB  algorithm
    for non-uniform mesh grids, by decoupling spatial and momentum space  discretizations  in  the  LB  scheme.  The  underlying
    momentum space discretization was the well-known \texttt{D2Q9} lattice, referred to in the study as `9-bit BGK model  in  2D
    space' and semantic variations thereof.

    Nine years after the debut of LB methods, a study~\cite{1997-HeX+LuoLS-PhysRevE} showed that they could be directly  derived
    from    the    continuous    Boltzmann    equation    with    linearized    collision     operator     under     the     BGK
    approximation~\cite{2011-HarrisS-Dover}, while lattice stencils  from  requirements  of  matching  continuous  and  discrete
    velocity moments up to a desired order---a decisive publication,  not  only  in  making  LB  methods  independent  from  its
    historical predecessor LGA, but also to pave the way towards modern methods  for  lattice  weights  determination  from  the
    lattice  velocity  set~\cite{2014-MattilaKK+PhilippiPC-SciWorldJ}.  The  lattices   in~\cite{1997-HeX+LuoLS-PhysRevE}   were
    verbosely referred to as `$d$-dimensional $b$-bit $g$ lattice model', with $d$ being the Euclidean space dimension, $b$  the
    lattice velocity count, and $g$ a geometry term, such as `triangular,' etc.

    A review article by Shen and  Doolen~\cite{1998-ChenS+DoolenGD-AnnuRevFluidMech}  published  a  decade  after  McNamara  and
    Zanetti's premiere LB publication~\cite{1988-McNamaraGR+ZanettiG-PhysRevLett} and seven years after Qian's paper introducing
    the now-prevailing `\texttt{DdQb}' lattice naming scheme~\cite{1991-QianYH+LallemandP-AdvKinTheoContMech}, would still refer
    to LB lattices either with LGA-style or verbose nomenclatures, and to  overall  LB  schemes  based  on  its  collision  term
    treatment,  such  as  `lattice  BGK  (LBGK)'  models,   after   Bhatnagar-Gross-Krook~\cite{1954-BhatnagarPL+KrookM-PhysRev,
    2003-LiboffRL-bookSpringer, 2011-HarrisS-Dover}.

    Higher-order lattices were proposed  in~\cite{1998-PavloP+VahalaL-PhysRevLett}  for  two-  and  three-dimensional  Euclidean
    spaces. They were referred to as `octagonal grid (17-bit),' and as `3D ``octagonal'' 53-bit' models, respectively, and  were
    isotropic up to the sixth-order. Since octagons are not space-filling, plane-tiling geometries, the proposed  lattices  were
    not of the Bravais type, meaning they impose a decoupling between the spatial and the  momentum  space  discretizations,  as
    with the non-uniform mesh~\cite{1996-HeX+DemboM-JComputPhys}, and the method has to  resort  to  interpolations,  which  was
    later shown to cause spurious numerical diffusion~\cite[p.~429]{2006-ShanX+ChenH-JFluidMech}.

    Other     lattice     namings     of     the     early-     and     mid-2000's     include     verbose,     spelled      out
    ones~\cite{2001-dHumieresD+LallemandP-PhysRevE,  2005-LuXY-IntJModPhysC};  a  `$D_dQ_b$'  variant  of  Qian's  \texttt{DdQb}
    scheme~\cite{2003-NourgalievRR+JosephD-IntJMulFlow};     a     `groupI'     to     `groupIV'     regular     2D      polygon
    variant~\cite{2003-WatariM+TsutaharaM-PhysRevE, 2007-WatariM-PhysA}; a `$b$  ($d$D)'  short  designation  for  an  otherwise
    verbose one~\cite{2006-ChikatamarlaSS+KarlinIL-PhysRevLett}; an explicit lattice units velocity list, such as `$\{0, \pm  1,
    \pm 3\}$', in~\cite{2006-ChikatamarlaSS+KarlinIV-PhysRevLett}; a `dodecahedron' and an `icosahedron' ones that were shown to
    be stable for supersonic thermal flows~\cite{2006-WatariM+TsutaharaM-PhysA, 2007-WatariM-PhysA}.
 
    %---------------------------------------------------------------------------------------------------------------------------
    \subsection{Higher-order era lattice designations}

    The year of 2006 is seemingly a landmark for multi-velocity, higher-order LB schemes---and  incidently  for  lattice  naming
    schemes---as   evidenced   by   the   appearance   of    two    key    publications,    namely    those    of    Shan    and
    co-authors~\cite{2006-ShanX+ChenH-JFluidMech} and of Philippi and co-authors~\cite{2006-PhilippiPC+SurmasR-PhysRevE}.

    A  sistematic  discretization   framework   for   the   Boltzmann   equation   was   proposed   by   Shan   and   co-authors
    in~\cite{2006-ShanX+ChenH-JFluidMech}.  From  Kinetic  theory~\cite{2011-HarrisS-Dover,   2003-LiboffRL-bookSpringer},   the
    authors pointed out that successive Chapman-Enskog approximations of the Boltzmann equation obtain the Euler, Navier-Stokes,
    Burnett, and higher-order macroscopic equations---meaning progressively higher-order moments  of  the  continuous  Boltzmann
    equation express progressively higher-order macroscopic thermohydrodynamic descriptions. Moreover, the authors  demonstrated
    that   projecting   the   Boltzmann   equation   onto   order-$N$   truncated   tensorial   Hermite   polynomial   expansion
    bases~\cite{1949-GradH-CommPureApplMath}, lead to discrete LB models of corresponding  order-$N$  moments,  since  resulting
    Hermite expansion coefficients correspond to the velocity moments up to the chosen order.

    In these author's discretization framework, the lattice is viewed as a Hermite expansion \emph{quadrature}, and  the  naming
    convention was defined in terms of three parameters, namely, an Euclidean space dimension $D$, a quadrature  velocity  count
    $d$, and an algebraic degree of precision $n$ encoded in an `$E_{D,n}^{d}$' naming scheme---an order-$N$  Hermite  expansion
    requires      a      quadrature      degree      $n      \geqslant      2N$.      Citing      Qian      and      co-authors'
    lattices~\cite{1992-QianYH+LallemandP-EurophysLett}, they established the following comparisons, which were off  only  by  a
    scaling factor: $D2Q9 \propto E_{2,5}^{9}$, $D3Q15 \propto E_{3,5}^{15}$, and $D3Q19 \propto E_{3,5}^{19}$.

    Additionally, they stablished that Gauss-Hermite quadratures of the Boltzmann equation yield LB  models  with  \emph{minimum
    velocity count} for given degree of precision and Euclidean spacial dimension, without, however, the  ability  to  predefine
    (choose) the discrete velocity abscisae,  which  apart  from  special  cases  fails  to  produce  a  space-filling,  Bravais
    lattice---recalling that for LB methods, this means lower memory requirements but decoupled spatial  and  momentum  `meshes'
    that require interpolations, thus introducing artifacts such as spurious numerical diffusion.

    In the Appendix of reference~\cite{2006-ShanX+ChenH-JFluidMech}, the authors include a brief discussion  with  some  missing
    details on deriving quadratures on predefined Cartesian abscissae, which is the main requirement for space-filling,  Bravais
    lattices for non-interpolating, exact advection LB schemes. Results for the space-filling $E_{2,7}^{17}$ and  $E_{3,7}^{39}$
    quadratures were listed among the ones obtained with Gauss-Hermite quadratures.

    % Missing details include (i) how to switch from rank-0 polynomial base to the tensorial Hermite one, (ii) and from scalar
    % argument ξ to a vectorial 𝛏 one.

    Tackling the aspects associated in deriving space-filling, Bravais  lattices  aiming  at  sufficiently  high  orders  as  to
    approach thermal hydrodynamic  transport  problems,  Philippi  and  co-authors~\cite{2006-PhilippiPC+SurmasR-PhysRevE}  have
    proposed a new \emph{Method of Prescribed Abscissas\/}, MPA, for obtaining lattice weight values  and  scaling  factor  from
    predefined lattice arrangements.

    Departing from the continuous Boltzmann equation, the derivation of discrete velocity sets, i.e., the lattice  vectors,  was
    considered as a quadrature problem, in the senses that (i)~weights $w_i$ for lattice velocities $\pmb{\xi}_i$ would be found
    so that moments of the discrete equilibrium mass  distribution  function  $f^{eq}_i$  would  exactly  match  its  continuous
    counterpart, and (ii)~to warrant even-ranked velocity tensor isotropy, which,  in  turn,  translates  into  isotropic  fluid
    transport properties.

    In the process, authors derived, according to their Eq.~\{6\}, that \emph{lattice velocity weights are functions of velocity
    magnitudes}; hence, of lattice energy levels. This outcome is central to the lattice naming scheme herein proposed.


    prescribed abscissas is described, along with a plethora of new, higher-order, multi-velocity lattices derived by it.

    focused on space-filling, Bravais lattices
    ...
    emphasis on microscopic velocity \emph{energy levels}
    ...

    % Seek for higher-order lattices (HOL)

    % HOL named in the DdQb scheme begin to display ambiguity

    % Reason: energy levels with similar velocity count

    % Look for velocity count integer sequences: square, hex 2D... 3D

